\section{Fazit}
Um das Zwischenspeichern der Simulationen zu ermöglichen, ist eine Kompressionsrate von Faktor $8-10$ notwendig. Die drei entwickelten Lösungsansätze  erreichten eine Kompressionsrate zwischen $11.6$ und $14.1$. Mit den Ansätzen ist das Zwischenspeichern der Simulationsdaten mit $70-85$ Megabyte Arbeitsspeicher realisierbar.\\
Mit den umgesetzten Kompressionen ist ein Streaming von $7-9$ Simulationen in der Sekunde möglich bei einer Internetverbindung von $5$ Megabit. Zum Vergleich: Im Ist-Zustand können bei der selben Internetverbindung $0.6$ Simulationen in der Sekunde heruntergeladen werden. Im Ist-Visualisierung werden im Allgemeinen $1$ bis Maximal $10$ Simulationen in der Sekunde benötigt. Durch die Kompressionen ist bei einer modernen Internetverbindung das on-the-fly Laden von komprimierten Simulationen möglich.
 
Der Lösungsansatz des Adaptives Subsamplings ist einfach umzusetzen und beinhaltet minimale Artefakte. Die Kompressionrate wurde erreicht, indem nur Daten übertragen werden, welche in der Visualisierung angezeigt werden. Wenn in Zukunft mehr Daten visualisiert werden, muss der JHelioviewer entweder eine Interpolation durchführen oder das Adaptive Subsampling muss mehr Daten übertragen, wobei die Kompressionsrate sinken wird.

Die Kompression mit der Diskreten Kosinus Transformation erreichte die höchste Kompressionsrate aller Lösungsansätze. Die Kompression fügt aber markante Artefakte ein, welche in der Visualisierung stören. Die Umsetzung der Dekompression gestaltet sich Komplex. Eine naive Implementation der inversen Kosinus Transformation führt zu einer Laufzeit, die um Faktor $100$ Langsamer ist als die anderen Lösungsansätze.
Die Artefakte können durch eine Glättung behoben werden. In der Visualierung des JHelioviewers sind auch bei diesem Ansatz kaum Artefakte sichtbar.\\
Es besteht die Möglichkeit, dass das Streaming von Simulationsdaten in einer höheren Kadenz verlangt werden. Die DCT Lösung kann in diesem Fall gut eingesetzt werden. Die Qualität sinkt langsam im Vergleich zur kompressionsrate. Dies ist möglich, es muss aber auf artefaktfreien Zoom verzichtet werden. Diese Vorgabe reduziert die mögliche Kompressionsrate deutlich, da der Benutzer selbst kleine Artefakte schnell entdecken kann.  Der Nachteil der DCT Lösung ist, dass die Dekompresssion rechenaufwendig ist.

Die Prediktive Kodierung erreichte eine vergleichbare Kompressionsrate wie die des DCT Lösungsansatzes. Die Kompressionsrate liegt mit $13.6$ nahe an der DCT Lösungsansatz welche eine Rate von $14.1$ erreichte. Die Kompression Prädiktive Kodierung beinhaltet schwächere Artefakte und werden vom menschlichen Auge schnell übersehen. Bei hoher Zoomstufe und bei Dichten Feldliniensituationen sind die Artefakte dennoch zu erkennen. Eine leichte Glättung kann aber alle Artefakte aus der Visualisierung verstecken.\\
Die Implementation Dekompression gestaltet sich im Vergleich zur DCT Lösung simpel. Es werden keine aufwändige Rückwärtstransformationen benöftigt, was sich auf die Laufzeit und die Komplexität der Implementation auswirkt.\\
Die Feldliniendaten werden von anderen Institutionen und Forschern verlangt. Es ist von Vorteil wenn die Dekompression mit moderaten Programmierkenntnissen umgesetzt werden kann. Die Implementation der Dekompression im JHelioviewer kann als Musterimplementation fungieren.\\
Der Lösungsansatz der Prädiktiven Kodierung wurde als finale Lösung ausgewählt. Es ist ein Kompromiss zwischen Kompression, Artefakte und Komplexität der Dekompression.\\

Das auftreten von Kompressionsartefakten ist momentan der limitierende Faktor für die Kompression: Im JHelioviewer sind auch leicht ausgeprägte Artefakte störend, da sie durch das Zoom Feature entdeckt werden können. Für weitere Forschung sind Post-Processing filter interessant, welche die Kompressionsartefakte dämpfen. Weiter interessant sind die Verfahren Wavelet Transformation, Compressive Sensing und Curve Fitting. DIe Wavelet Transformation ist je nach Wavelet weniger anfällig auf Ringing artefakte. Compressive Sensing lässt sich auf den Anwendungsfall stark spezialisieren. Curve Fitting ist ebenfalls ein mögliches Verfahren. die Artefakte einer Bildkompression mit Curve Fitting sind Rauschunterdrückung und allgemein schärfung des Bildes. 