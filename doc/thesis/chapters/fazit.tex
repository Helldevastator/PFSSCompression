\section{Fazit}
Die Ist-Kompression benötigt durchschnittlich 1 Megabyte an Daten pro Simulation der Feldlinien. Das Caching von $1000$ Simulationen verbraucht ein Gigabyte an Arbeitsspeicher. Da der Arbeitsspeicher ebenfalls für die Zwischenspeicherung weiterer Daten benötigt wird, ist diese Datenmenge nicht vertretbar. Um das Zwischenspeichern der Simulationen zu ermöglichen, ist eine Kompressionsrate von Faktor $8-10$ notwendig. Die entwickelten Kompressionsverfahren Adaptives Subsampling, DCT Kompression und Prädiktive Kodierung erreichten eine Kompressionsrate von $11.6$, $14.1$ und $13.6$. Das Zwischenspeichern von $1000$ Simulationen benötigt zwischen $70$ und $85$ Megabyte Arbeitsspeicher. Somit ist das Caching mit allen entwickelten Kompressionsverfahren realisierbar.

Ein Forschungsziel ist, unter welchen Bedingungen ein Streaming der Simulationen möglich ist. Es wird angenommen, dass für die Feldlinien $5$ Megabit Bandbreite zur Verfügung stehen. Mit dieser Bandbreite können im Ist-Zustand $0.6$ Simulationen in der Sekunde heruntergeladen werden. Der JHelioviewers benötigt im Allgemeinen $1$ bis maximal $10$ Simulationen in der Sekunde für die Visualisierung. Mit derselben Bandbreite erreichen die entwickelten Kompressionen durchschnittlich $7$, $9$ und $8$ Simulationen in der Sekunde. Der Maximalfall von $10$ Simulationen in der Sekunde benötigt mehr Bandbreite oder eine höhere Kompression zu schlechterer Qualität. Für den allgemeinen Fall ist mit den entwickelten Kompressionen und einer modernen Internetverbindung das Streaming möglich. 

Das Auftreten von Ringing oder Ringing ähnlichen Artefakten ist der limitierende Faktor für die entwickelten Kompression: Im JHelioviewer sind auch leicht ausgeprägte Ringing Artefakte störend, da sie durch das Zoom Feature entdeckt werden können. Wenn für das Fernziel, eine flüssige Animation der Feldlinien die Anzahl übertragenen Simulationen pro Sekunde erhöht wird, muss auf den artefaktfreien Zoom verzichtet werden.

Das Verfahren des Adaptiven Subsamplings ist einfach umzusetzen und beinhaltet minimale Artefakte. Es werden ausschliesslich die Daten übertragen, welche der JHelioviewer für die Visualisierung benötigt. Die Datenmenge ist begrenzt durch die Leistung der Grafikkarte. Wenn in Zukunft die zu visualisierenden Datenmenge erhöht wird, sinkt die Kompressionsrate des Verfahrens. Das Problem weisen die Kompressionsverfahren DCT und Prädiktive Kodierung nicht auf, da mehr Daten übertragen werden, als die Visualisierung benötigt.

Die Kompression mit der Diskreten Kosinus Transformation erreichte die höchste Kompressionsrate aller Verfahren. Die Kompression fügt Ringing Artefakte ein, welche in der Visualisierung stören. Die Artefakte können durch eine Glättung behoben werden. Die Umsetzung der Dekompression gestaltet sich Komplex: Eine naive Implementation der inversen Kosinus Transformation führt zu einer Laufzeit, die um Faktor $100$ Langsamer ist als eine naive Implementation der anderen Kompressionsverfahren. Forschungsinstitutionen wie Beispielsweise IRAP \cite{website:irap} sind an Feldliniensimulationen interessiert. Es ist von Vorteil wenn die Dekompression mit moderaten Programmierkenntnissen umgesetzt werden kann. Die Dekompression dieses Verfahrens beinhaltet die grösste Komplexität.

Die Prädiktive Kodierung erreichte eine vergleichbare Kompressionsrate wie die des DCT Verfahrens. Die Artefakte sind weniger Ausgeprägt und erst bei hohen Zoomstufen erkennbar. Mit einer leichten Glättung sind die Artefakte in der Visualisierung nicht mehr zu erkennen. Das Verfahren verwendet keine komplexen Transformationen: Wenn Institutionen wie Beispielsweise IRAP \cite{website:irap} eine Dekompression entwickelt, sind für die Umsetzung weniger Programmierkenntnisse notwendig als beim DCT-Verfahren. Das Verfahren der Prädiktiven Kodierung wurde als finale Lösung ausgewählt. Es ist ein Kompromiss zwischen Kompression, Artefaktbildung und Komplexität.

Das Kompressionsverfahren der Prädiktiven Kodierung kann für anderen Anwendungsfälle eingesetzt werden wie Kompression von Feldlinien von Spulen oder Flugbahnen der Teilchenphysik. Das entwickelte Dateiformat ist auf kein Koordinatensystem oder maximale Genauigkeit begrenzt. Die Quantisierung muss auf den jeweiligen Anwendungsfall angepasst werden : Es ist ein Kompromiss zwischen Kompressionsrate und Artefakte, welche für den jeweiligen Fall entschieden werden muss. Eine ähnliche Kompressionsrate kann erwartet werden, wenn die Daten mit 16 Bit Genauigkeit abgespeichert werden und die Punktfolgen niederfrequenten Schwingungen beinhalten.

Für weitere Forschungen sind Kompressionsverfahren interessant, welche kaum oder keine Ringing Artefakte einfügen wie die Wavelet Transformation, Compressive Sensing und Curve Fitting. Die Artefakte der Wavelet Transformation kann durch Auswahl des Wavelets beeinflusst werden. Compressive Sensing kann auf die zu komprimierenden Daten optimiert werden. Es kann die Eigenschaft genutzt werden, dass die Feldlinien sich hauptsächlich in Skalierung, Rotation und Verschiebung unterscheiden. Curve Fitting ist ebenfalls ein mögliches Verfahren. Die Artefakte einer Bildkompression mit Curve Fitting sind Rauschunterdrückung und Kantenschärfung des Bildes. Ähnliche Kompressionsartefakte können entstehen, wenn das Verfahren Feldliniendaten angewendet wird.


