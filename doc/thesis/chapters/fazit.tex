\section{Fazit}
Die Ist-Kompression benötigt durchschnittlich 1 Megabyte an Daten pro Simulation der Feldlinien. Das Caching von 1000 Simulationen verbraucht 1 Gigabyte an Arbeitsspeicher. Da der Arbeitsspeicher ebenfalls für die Zwischenspeicherung weiterer Daten benötigt wird, ist diese Datenmenge nicht vertretbar. Um das Zwischenspeichern der Simulationen zu ermöglichen, ist eine Kompressionsrate von Faktor 8 bis 10 notwendig. Die entwickelten Kompressionsverfahren Adaptives Subsampling, DCT Kompression und Prädiktive Kodierung erreichten eine Kompressionsrate von 11.6, 14.1 und 13.6. Das Zwischenspeichern von 1000 Simulationen benötigt zwischen 70 und 85 Megabyte Arbeitsspeicher. Das Caching der Feldlinien kann mit den entwickelten Kompressionsverfahren speichereffizient umgesetzt werden.

Weiterhin wurde das Streaming der Feldlinien-Simulationen erforscht: Mit derselben Bandbreite können im Ist-Zustand 0.6 Simulationen in der Sekunde heruntergeladen werden. Der JHelioviewers benötigt bis zu 10 Simulationen in der Sekunde für die Visualisierung. Die entwickelten Kompressionsverfahren erreichen einen Durchsatz von durchschnittlich 8 Simulationen in der Sekunde. Für den allgemeinen Anwendungsfall ist mit den entwickelten Kompressionen und einer modernen Internetverbindung das Streaming möglich.

Das Kompressionsverfahren mittels Prädiktiver Kodierung wurde als finale Kompression ausgewählt. Das Verfahren erreichte die höchste Qualität zur zweitbesten Kompressionsrate. Die Artefakte der Kompression wirken sich meistens als Verschiebungen aus. In der Visualisierung sind solche Artefakte für das menschliche Auge schwieriger zu erkennen, als die Artefakte der DCT Kompression. Gegenüber der DCT Kompression ist die Prädiktive Kodierung einfach zu implementieren: Es werden keine komplexen Transformationen verwendet, eine Implementation der Dekompression ist weniger anspruchsvoll. Die Feldliniendaten sind öffentlich zugänglich, eine einfache Implementation vereinfacht Dritten wie Beispielsweise IRAP \cite{website:irap} den Zugriff. Das Verfahren mittels Adaptives Subsampling ist ähnlich wie die Prädiktive Kodierung einfach zu implementieren und das menschliche Auge kann die Kompressionsartefakte schwerer erkennen, als die der DCT Kompression. Die Prädiktive Kodierung erreichte die höhere Kompression zu einer höheren Qualität als das Verfahren des Adaptiven Subsamplings. Das Verfahren der Prädiktiven Kompression ist ein Kompromiss zwischen Kompressionsrate, Qualität und Komplexität der Implementation.

Das Verfahren mittels Prädiktiven Kodierung kann für anderen Anwendungsfälle eingesetzt werden wie Kompression von Feldlinien von Spulen oder Flugbahnen der Teilchenphysik. Das entwickelte Dateiformat ist auf kein Koordinatensystem oder maximale Genauigkeit begrenzt. Für eine optimale Kompression muss die Quantisierung auf den jeweiligen Anwendungsfall angepasst werden.

Für weitere Forschungen sind die Verfahren Wavelet-Transformation, Compressive Sensing und Curve Fitting interessant. Die Wavelet-Transformation ist stark mit der Kosinus-Transformation verwandt. Die Wahl des Wavelets kann die Kompression und die Artefaktbildung positiv beeinflussen. Die Wavelet-Transformation hat das Potential eine ähnliche Kompressionsrate wie die DCT Kompression zu erreichen zu einer höheren Qualität.\\
Compressive Sensing repräsentiert ein Signal durch eine minimale Anzahl an Funktionen (sparse representation), welche in einem Dictionary definiert sind. Die Feldlinien unterscheiden sich am stärksten in Rotation, Verschiebung und Skalierung. Die Formen der Feldlinien sind aber meist ähnlich. Mit einem geeigneten Dictionary können die Feldlinien durch wenige Funktionen dargestellt werden und das Verfahren erreicht eine hohe Kompression zu einer hohen Qualität. Es existieren Algorithmen für die Berechnung der sparse representation und für das Erstellen des Dictionary's. Die Algorithmen sind aber im Vergleich zu Kosinus- oder Wavelet Transformation komplex zu implementieren. Für eine Datenkompression müssen Detailprobleme, wie die Übertragung des Dictionary's gelöst werden. Die sparse representation ist Rechenintensiv und die Laufzeit der Kompression wird deutlich höher sein als bei Verfahren mittels Prädiktiver Kodierung.\\
Curve Fitting wird für Signalinterpolation oder Rauschunterdrückung verwendet. Es ist aber eine Datenkompression möglich. Der Vorteil dieses Verfahrens liegt in der Artefaktbildung: Die Artefakte einer Kompression mittels Curve Fitting drücken sich als Glättung aus und sind für das menschliche Auge schwieriger zu erkennen. In der Datenkompression ist Curve Fitting kein etabliertes Verfahren. Es existieren hauptsächlich Machbarkeitsnachweise, aber ein Kompressionsstandard mit Curve Fitting ist nicht bekannt. Es ist ebenfalls schwierig performante Programmbibliotheken, welche Curve Fitting Algorithmen anbietet, aufzufinden. Die Entwicklung einer Kompression mittels Curve Fitting zieht im Vergleich zu etablierten Kompressionsverfahren wie DCT zusätzlichen Aufwand mit sich.