\section{Eigenschaften und Kompression der Feldlinien}
Jede Simulation beinhaltet 1200 Feldlinien und insgesamt etwa 60'000 Punkte. Fits format \cite{website:fits}

\subsection{Ist-Komprimierung} \label{konzept:ist-komprimierung}
Der JHelioviewer bietet an, die Feldlinien zu einem gegebenen Zeitpunkt darzustellen. Damit der Benutzer eine vernünftige Zeit auf die Feldlinien wartet, wurde bereits im Vorfeld eine Kompression implementiert. Zuerst werden die Daten im sphärischen Koordinatensystem(Radius, Längengrad $\phi$ und Breitengrad $\theta$) auf dem Server quantisiert und mit GZip verlustfrei komprimiert. Der JHelioviewer dekomprimiert die Daten und transformiert sie in das kartesische Koordinatensystem um. Die Punktmenge wäre für schwächere Grafikkarten zu gross, weshalb der JHelioviewer eine weitere Quantisierung durchführt.\\
[\baselineskip]
\textbf{Quantisierung und Dateiformat auf dem Server}\\
Zuerst werden die Kanäle R,$\phi$ und $\theta$ Kanäle zu shorts diskretisiert:
\begin{enumerate}
 \item R: 4 = $2^{15}$. 
 \item $\phi$: $2\pi$ = $2^{15}$
 \item $\theta$: $2\pi$ = $2^{15}$
\end{enumerate}
$\theta$ Wertebereich geht aber nur von 0 bis $\pi$, die letzten Bits werden gar nicht verwendet. Die Kanäle R und $\phi$ haben das Problem, dass der Wert $2^{15}$ einen Signed Integer Overflow verursacht und auf $-2^{15}$ zu liegen kommt. R scheint den maximalen Wert nie zu erreichen. Wenn aber eine Feldlinie den Nullpunkt passiert, springt der Kanal von  $2^{15}-1$ auf $-2^{15}$ und dann auf 0.\\
[\baselineskip]
Subsampling, jeder vierte Punkt 
0 Löschen.

Format: zuerst Konstanten, alle Radien\\
[\baselineskip]
\textbf{Quantisierung des JHelioviewers}\\
Clientseitig wieder ein subsampling und umrechnung ins kartesische System xyz, dann Anglesubsampling
	
\subsection{Lösung 0, Angle-Subsampling}
Subsampling 5Grad auf dem Server
Minimale Lösung, ziel ist es diese Lösungen zu schlagen.

\subsection{Lösung 1, Diskrete Kosinus Transformation}
DCT, da alles nahe an harmonischen Halbwellen

subsampling
koordinatentransformation --> kein wrap around,
Ableitung
Cosinus-Transformation

DCT 2
idct ist dct3

Quantisierung

speicherung für Entropie encoding, alles was ähnlich ist zusammen.

encoding-> rar



