\section{Eigenschaften und Kompression der Feldlinien} \label{konzept}
Jede Simulation beinhaltet 1200 Feldlinien und insgesamt etwa 60'000 Punkte. Fits format \cite{website:fits}

\subsection{Ist-Komprimierung} \label{konzept:ist-komprimierung}
Der JHelioviewer bietet an, die Feldlinien zu einem gegebenen Zeitpunkt darzustellen. Damit der Benutzer eine vernünftige Zeit auf die Feldlinien wartet, wurde bereits im Vorfeld eine Kompression implementiert. Zuerst werden die Daten im sphärischen Koordinatensystem(Radius, Längengrad $\phi$ und Breitengrad $\theta$) auf dem Server quantisiert und mit GZip verlustfrei komprimiert. Der JHelioviewer dekomprimiert die Daten und transformiert sie in das kartesische Koordinatensystem um. Die Punktmenge wäre für schwächere Grafikkarten zu gross, weshalb der JHelioviewer eine weitere Quantisierung durchführt.\\
[\baselineskip]
\textbf{Quantisierung und Dateiformat auf dem Server}\\
Zuerst werden die Kanäle R,$\phi$ und $\theta$ Kanäle zu shorts diskretisiert:
\begin{enumerate}
 \item R: 4 = $2^{15}$. 
 \item $\phi$: $2\pi$ = $2^{15}$
 \item $\theta$: $2\pi$ = $2^{15}$
\end{enumerate}
$\theta$ Wertebereich geht aber nur von 0 bis $\pi$, die letzten Bits werden gar nicht verwendet. Die Kanäle R und $\phi$ haben das Problem, dass der Wert $2^{15}$ einen Signed Integer Overflow verursacht und auf $-2^{15}$ zu liegen kommt. R scheint den maximalen Wert nie zu erreichen. Wenn aber eine Feldlinie den Nullpunkt passiert, springt der Kanal von  $2^{15}-1$ auf $-2^{15}$ und dann auf 0.\\
[\baselineskip]
Subsampling, jeder vierte Punkt 
0 Löschen.

Format: zuerst Konstanten, alle Radien\\
[\baselineskip]
\textbf{Quantisierung des JHelioviewers}\\
Clientseitig wieder ein subsampling und umrechnung ins kartesische System xyz, dann Anglesubsampling
	
\subsection{Lösung 0, Angle-Subsampling}
Subsampling 5Grad auf dem Server
Kodierung mittels rar
\subsubsection{Subsampling}\label{konzept:loesung0:subsampling}

\subsubsection{Ablegung ins Fits Format} \label{konzept:loesung0:fits}

\subsubsection{Entropie Kodierung} \label{konzept:loesung0:kodierung}
Rar hat sich bewährt bei der purer Feldlinien kompression im Vergleich zu anderen Verfahren wie LZ77/gZip

\subsection{Lösung 1, Diskrete Kosinus Transformation}
DCT, da alles nahe an harmonischen Halbwellen
kartesische Koordinaten --> kein wrap around,

Es ist auch möglich die Punkte im sphärischen Koordinatensystem in den Frequenzraum zu überführen. Der $\phi$-Kanal ist jedoch schwierig durch tiefe Kosinus Schwingungen darzustellen: Wie im Abschnitt \ref{konzept:ist-komprimierung} besprochen, beinhaltet der Kanal Sprünge bei der Passierung des Nullpunktes. Das führt zu sehr hochfrequenten Schwingungsanteile in der DCT. Nach einer Quantisierung sind dabei Artefakte nicht vermeidbar. Im kartesischen System hingegen sind alle Kanäle stetig und lassen sich einfacher durch Kosinus-Funktionen approximieren.\\

\subsubsection{Subsampling} \label{konzept:loesung1:subsampling}
Wie im Abschnitt \ref{testsetup:auswahl_erhebung} beschrieben, wurde aus den Testdaten die Quantisierung und das Subsampling entfernt. Die Feldlinien der Testdaten haben das vier Mal mehr Puntke, als beim Ist-Zustand übertragen werden. Die DCT-Implementierung weist eine Komplexität von $O(n^2)$ auf. Vor der Kosinus-Transformation wird deshalb das selbe Subsampling durchgeführt, wie im Ist-Zustand. So kann der Rechenaufwand in Grenzen gehalten werden.\\
Falls die Laufzeit der Dekompression verbessert werden soll, kann die Fast-Cosine-Transformation umgesetzt werden. Diese hat eine Komplexität von $O(n log(n))$. Der Nachteil ist, dass nur Daten der Länge $2^n$ transformiert werden können, was zusätzliche Programmlogik braucht. Falls die Fast-Cosine-Transformation nicht ausreicht, können die Linien in Blöcken mit einer bestimmten Anzahl von Punkten unterteilt werden. Dadurch wird die Komplexität auf $O(n)$ gesenkt. Jedoch ist es wahrscheinlich, dass durch die Unterteilung die Kompressionsrate leidet. Vermutlich braucht es für die Approximation der Blöcke insgesamt mehr Kosinus-Funktionen, als für die Approximation der gesamten Feldlinie.\\

\subsubsection{Randbehandlung} \label{konzept:loesung1:randbehandlung}

\subsubsection{Cosinus-Transformation} \label{konzept:loesung1:kosinus}
Cosinus-Transformation
	DCT 2
	idct ist dct3
	
\subsubsection{Quantisierung}

\subsubsection{Ablegung ins Fits Format}

\subsubsection{Entropie Kodierung}\label{konzept:loesung1:kodierung}
Kodierung
	Zuerst einfaches Run-length

	Weitere Kodierung mittels Rar.
		Rar hat sich bewährt bei der purer Feldlinien kompression im Vergleich zu anderen Verfahren wie LZ77/gZip





