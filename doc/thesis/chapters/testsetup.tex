\section{Testsetup}
Verlustbehaftete Kompression muss die Genauigkeit gewährleistet werden.
Es kann immer eine höhere Kompression unter dem Kompromiss der Genauigkeit.
Eine feste Grenze für die genauigkeit ist nicht immer festzulegen.
Um Verfahren trotzdem zu Testen und zu vergleichen werden kompressionen mehrfach in verschiedenen qualitätsstufen getestet und jeweils den Fehler und die resultierende Dateigrösse verglichen.

Bild eines Beispielgraphen

\subsection{Auswahl und Erhebung der Testdaten}
Testdaten sollen zu einem Randfälle abdecken, als auch durchschnittliche Fälle enthalten.
Insgesamt wurden 10 Datensätze ausgewählt. Nach grossen Solar Flares gesucht vier Datensätze mit grossen Flares, zwei mit sehr wenig Sonnenaktivität und vier zufällig.
Die feldlinien werden aber nur alle sechs Stunden berechnet und ein Flare ist ein kurzes ereignis. Es wurden die Datensätze vor dem Ereignis ausgewählt.

wie im Abschnitt \ref{konzept:ist-komprimierung} beschrieben, führt der IDL-Code schon eine Quantisierung durch. deshalb wurde der IDL Code angepasst, quantisierung entfernt und dateien mit float genauigkeit genommen.  

Problem mit letzter linie

\subsection{Messung des Fehlers}
Expected

Actual

Acutal <= expected

Ich weiss, welches der Originalpunkt ist.
