\section*{Abstract}
Ziel dieser Arbeit ist es ein verlustbehaftetes Kompressionverfahren von wissenschaftlichen Daten zu entwickeln, welche die Übertragung und Zwischenspeicherung der Daten ermöglicht.
In dieser Arbeit wurden drei Verfahren entwickelt: Kompression mit Adaptivem Subsampling, Kompression mit einer Diskreten Kosinus Transformation und Kompression mit Prediktoren. Die höchste Kompressionsrate wurde mit der Diskreteten Kosinus Transformation erreicht, während die Kompression mit Adaptiven Subsamplings keine sichtbaren Kompressionsartefakte aufweist. Die Kompression mit Prediktoren ist ein Kompromiss zwischen Kompressionsrate und Artefakte.

Mit den entwickelten Verfahren ist die Übertragung von wissenschaftlichen Daten in vernünfiger Zeit ?umsetzbar?

Ziele wurden erreicht?

Weitere Möglichkeiten ist ein Kompressionsverfahren mit Wavelet Transformation oder Compressive Sensing.