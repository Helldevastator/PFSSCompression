\section*{Abstract}
JHelioviewer ist eine Applikation, welche Mess- und Simulationsdaten der Sonne visualisiert. Die Daten für die Visualisierung werden zur Laufzeit über eine Internetverbindung geladen. Ziel dieser Arbeit ist es eine verlustbehaftete Kompression für wissenschaftliche Simulationen zu entwickeln, welche die Übertragung und Zwischenspeicherung ermöglicht.

In dieser Arbeit wurden drei Verfahren entwickelt: eine Kompression mit Adaptivem Subsampling, eine Kompression mit einer Diskreten Kosinus Transformation und eine Kompression mit Prädiktoren. Die höchste Kompressionsrate wurde mit der Diskreteten Kosinus Transformation erreicht, während die Kompression mit Adaptiven Subsamplings minimale Kompressionsartefakte aufweist. Die Kompression mit Prädiktoren ist ein Kompromiss zwischen Kompressionsrate und Artefakte.

Die Übertragung und Zwischenspeicherung von wissenschaftlichen ist mit den entwickelten Verfahren möglich. Der Limitierende Faktor für die Kompression ist die Artefaktbildung: Ringing oder Ringing-Ähnliche Artefakte sind in der JHelio-Visualisierung störend. Durch den Zoom sind schwach ausgeprägte Artefakte zu erkennen. Eine höhere Kompressionsrate ist mit Verfahren zu erreichen, welche kaum oder keine Ringing Artefakte produzieren wie Wavelet Transformation, Curve Fitting oder Compressive Sensing.