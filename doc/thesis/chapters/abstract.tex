\section*{Abstract}
Die Darstellung von Feldlinien im dreidimensionalen Raum benötigen eine hohe Datenmenge. Ziel dieser Arbeit ist es eine verlustbehaftete Kompression für Feldlinien zu entwickeln, welche eine schnelle Übertragung und effizientes Caching ermöglicht. Im Vorfeld wurde ein Kompressionsverfahren entwickelt, die die Datenmenge um einen Faktor 10 komprimiert. Die komprimierte Datenmenge ist dennoch zu gross für eine performante Übertragung. In dieser Arbeit wurden drei Kompressionsverfahren entwickelt, welche die Effizienz der Übertragung und des Cachings verbessern. Mit einer Prädiktiven Kodierung konnte einen Kompressionsfaktor von 13.6  gegenüber der Ist-Kompression erreicht werden zu einer vergleichbaren Qualität. Die Kompressionsartefakte der Prädiktiven Kodierung wirken sich meist als Verschiebung der Feldlinie aus, während die Form der Feldlinie erhalten bleibt. In einer Visualisierung sind die Artefakte dieser Kompression für das menschliche Auge schwieriger zu erkennen als die Artefakte der anderen entwickelten Verfahren.