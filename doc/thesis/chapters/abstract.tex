\section*{Abstract}
Ziel dieser Arbeit ist es ein verlustbehaftetes Kompressionverfahren von wissenschaftlichen Daten zu entwickeln, welche die Übertragung und Zwischenspeicherung der Daten ermöglicht.

In dieser Arbeit wurden drei Verfahren entwickelt: Kompression mit Adaptives Subsampling, Kompression mit einer Diskreten Kosinus Transformation und Kompression mit Prediktoren. Die höchste Kompressionsrate wurde mit der Diskreteten Kosinus Transformation erreicht, während die Kompression mit Adaptiven Subsamplings keine sichtbaren Kompressionsartefakte aufweist. Die Kompression mit Prediktoren ist ein Kompromiss zwischen Kompressionsrate und Artefakte.


Die Kompressionsrate des Adaptiven Subsampling Verfahrens beträgt $11.6$. Die DCT Kompression 
Variante Adaptives Subsampling, DCT und Prediktive Kodierung.
Adaptives Subsampling erreicht eine Kompressionsrate von $11.6$, indem es einen grossteil der Informationen verwirft.
Die DCT Variante erreicht eine Kompressionsrate von $14.1$. Dieser Ansatz führt Artefakte ein, welche in der Visualisierung als störend empfunden werden.
Die Prediktive Kodierung erreicht eine Kompressionsrate von $12.6$. Es ist beinahe Artefaktfrei.
Die Dekompression wurde erweitert?

Weiterführende forschung?